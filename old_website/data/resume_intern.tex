% LaTeX file for resume 
% This file uses the resume document class (res.cls)

\documentclass{res}
\RequirePackage{marvosym}
\usepackage{color} 
\usepackage{xspace}
\usepackage{hyperref}
\usepackage{comment}


\setlength{\textheight}{9in} % increase text height to fit on 1-page 
\newcommand{\latex}{\LaTeX\xspace}

\begin{document} 

 %\moveleft.5\hoffset\centerline{\LARGE\bf Hirak Sarkar}
 \moveleft\hoffset\leftline{\LARGE\bf Hirak Sarkar}
% \name{HIRAK SARKAR\\[12pt]}     % the \\[12pt] adds a blank
				        % line after name  
%\moveleft\hoffset\vbox{\hrule width\resumewidth height 1pt}

\moveleft\hoffset\vbox{\hrule width\resumewidth height 0.7pt}    
\vspace{-0.5cm}
\address{8 Acorn Ln\\Stony Brook, NY-11790\\Mob No.+1 6315208131}
%\Letter \hspace{2mm} aychakrabort@cs.stonybrook.edu
\address{ \Letter \hspace{2mm}\href{mailto:hsarkar@cs.stonybrook.edu}{hsarkar@cs.stonybrook.edu}  \\ {\Large\ComputerMouse} \hspace{0.6mm}  \href{www.hiraksarkar.com}{www.hiraksarkar.com}}  
                                  
\begin{resume}
\section{Research Interest}
% \vspace{0.1in}
I am interested in applying machine learning techniques such as statistical inference and deep learning to analyse and extract information from  big data in the field of genomics, social and computer networks.

\vspace{-0.2in}
\section{Education}          
  \vspace{-0.1in}	
   \begin{tabbing}
   \hspace{2.3in}\= \hspace{2.6in}\= \kill % set up two tab positions
     {\bf Ph.D in Computer Science} \>   \>  2014-2019
   \end{tabbing}  \vspace{-20pt}      % suppress blank line after tabbing
  % {\bf Bachelor of Technology(Computer Science and Engineering)}  \\
       Statistical Inference in Biological Data, {\it advisor: Prof. Rob Patro}  \\        
       Stony Brook University, NY     \\
       GPA: 3.99/4.00
 
 
 \vspace{-0.1in}	
   \begin{tabbing}
   \hspace{2.3in}\= \hspace{2.6in}\= \kill % set up two tab positions
     {\bf Masters of Technology (Computer Science)}  \>     \>2011-2013 
   \end{tabbing}  \vspace{-20pt}      % suppress blank line after tabbing
  % {\bf Bachelor of Technology(Computer Science and Engineering)}  \\        
       Indian Statistical Institute     \\
       $1^{st}$ Class (Hons.) 


\vspace{-0.1in}	
\begin{tabbing}
\hspace{2.3in}\= \hspace{2.6in}\= \kill % set up two tab positions
{\bf Bachelor of Technology (Computer Science and Engineering)}  \>     \>2007-2011
\end{tabbing}  \vspace{-20pt}      % suppress blank line after tabbing
  % {\bf Bachelor of Technology(Computer Science and Engineering)}  \\        
West Bengal University of Technology     \\       
GPA: 8.88/10      \\   
%B.Tech Thesis Topic: {\bf \color{blue} \underline {GameSAT: A Structured Approach to Combine SLS SAT Solvers}}  \\

\vspace{-0.9cm}
\section{Publications}
\begin{enumerate}

\item {\it An Efficient, Scalable and Exact Representation of High-Dimensional Color Information Enabled via de Bruijn Graph Search}, by Hirak Sarkar, Avi Srivastava and Rob Patro [\textit{\textbf{ISMB'18}]. 


\item {\it An Efficient, Scalable and Exact Representation of High-Dimensional Color Information Enabled via de Bruijn Graph Search}, by Fatemeh Almodaresi*, \underline{Hirak Sarkar*}, Avi Srivastava and Rob Patro [\textit{\textbf{ISMB'18}]. 

\item   {\it Quark enables semi-reference-based compression of RNA-seq data} by  \underline{Hirak Sarkar} and Rob Patro [\textit{accepted \textbf {Bioinformatics'17}, impact factor: 7.307}].


\item   {\it Fast, Lightweight Clustering of de novo Transcriptomes using Fragment Equivalence Classes} by A Srivastava*, \underline{Hirak Sarkar*}, Laraib Malik and Rob Patro (* \textit{Joint first authors}) [\textit{\textbf{RECOMB-seq'16}}]. 


\item {\it RapMap: A Rapid, Sensitive and Accurate Tool for Mapping RNA-seq Reads to Transcriptomes} by A Srivastava, \underline{Hirak Sarkar}, Nitish Gupta and Rob Patro  [\textit{\textbf{ISMB'16}, acceptance rate: 17\%}].

\item {\it Pufferfish: A fast graph-based indexing and query strategy for large genomic sequences} by Fatemeh Almodaresi*, \underline{Hirak Sarkar*}, and Rob Patro, Poster presented in [\textit{\textbf{WABI'17}}].

\item {\it Joint probabilistic model for multiple steps of gene regulation} by \underline{Hirak Sarkar}, Yi-Fei Huang and Adam Siepel, Poster presented in  \textit{\textbf{BioData'16}}

\item {\it Voronoi Game on Graphs} (Extended version) by S. Bandyapadhyay, A. Banik, S. Das and \underline{H. Sarkar} (in alphabetical order of surnames) {\it Journal of Theoretical Computer Science}  [\textit{\textbf{TCS'15}}].

\item {\it Voronoi Game on Graphs} by  S. Bandyapadhyay, A. Banik, S. Das and \underline{H. Sarkar} (in alphabetical order of surnames) Seventh International Workshop on Algorithms and Computation. \textit{\textbf{WALCOM'13}}.


\vspace{-0.5cm}
\end{enumerate}
\section{Professional Experience}
\begin{itemize}
\item {\textbf {Simons Center for Quantitative Biology, Cold Spring Harbor Lab:}} Worked under the supervision of Prof. Adam Seipel from May, 2016 to July, 2016. We designed probabilistic graphical model to infer transcription and degradation rates from different assays such as GRO-seq and RNA-seq.  

\item {\it Summer Assistantship '15,'17} with Prof. Rob Patro from May, 2015 to July, 2015. We worked on various problems ranging from 
\item  Teaching Assistant for CSE549 (Computational Biology), CSE219 (Game Programming)
\item {\it Visiting Researcher} at Advanced Computing \& Microelectronics Unit, Indian Statistical Institute from October, 2013 to December, 
2013. I worked on Computational Geometry and Graph Theory
\item {\it Junior Research Fellow} in Department of Computer Science \& Engineering at Indian Institute of Technology, Kharagpur (IIT) 
from July, 2013 to Sept, 2013. I was a member of Complex Network Engineering Group. I did TA-ship for Introductory Programming 
Course in that brief stint. 
\end{itemize}

\section{Relevant Course Projects}
\begin{itemize}

 \item{ \it{IPID Header Survey:}} We used IPID headers to estimate the load over different servers, sampled from alexa
 list. The main challenge of the project is to detect the wrapping pattern and navigate through the global vs local IPID counter. 
 We also looked at the temporal pattern of network traffic for the different regional websites which shows interesting correlation with possible 
 working load at the server end. \\
 {\it Instructor: Prof. Phillipa Gill}
 
 \item {{\it Some Geometric and Combinatorial Properties of Binary Matrices Related to
Discrete Tomography:}} Here we are trying to decompose an image matrix into matrices
each having orthogonal convex polygon also known as Ferrer?s digraph. An operation could
regenerate the original image from these matrices. The methods can be applied to image and
data compression. (\href{https://drive.google.com/open?id=0B3ErYrn4jOcxSHhReVF6cmROelE}{\it  \underline{Masters dissertation}}) 
{\it Advisor: Prof. Bhargab B Bhattacharya \& Prof. Sandip Das}

 \item 
 {{\it GameSAT- A Structured Approach to Combine SLS SAT Solvers:}} Here we used several existing heuristic algorithms to mix up with each other in a customized probability to solve combinatorial hard problems encoded as SAT problems. We used UBCSAT framework for experimentation. \\
({\it B.Tech dissertation}) {\it Advisor: Ashiqur KhudaBukhsh, CMU}
 
 \end{itemize}
 
 
\section{Awards and Honors}
   \begin{itemize}
   \item Awarded {\it Research Assistantship}, SBU ({\it 2016-present})
   \item Awarded {\it Special CS Chair Fellowship} ({\it of \$10000} ), SBU ({\it 2014-2015})
   \item Awarded {\it NUS Research Scholarship}, NUS ({\it Jan'14-June'14})
   \item Awarded {\it Post-graduate Scholarship} by, Govt. of India. ({\it 2011-2013}) 
   \item Received {\color{blue} First Prize} for Software Competition (IEM), Calcutta.
   \end{itemize}

\section{Programming Skills}
Python, C++,C

\section{\bf Open Source Tools Used}
Dendropy, BioNet  (Comp Bio) \\
NLTK, Scrapy, Scikit-learn, Stanford Parser, Pandas (Data Science) \\



\section{Relevant Coursework} 
\begin{itemize}
\item Artificial Intelligence,  Computational Biology, Analysis of Algorithms, Fundamental of Networks. (at {\it SBU})
\item Machine Learning \& Pattern Recognition, Image Processing, Stochastic Process, Optimization Algorithms, Computer Graphics. (at {\it
Indian Statistical Institute})
\end{itemize}



\end{resume}
\end{document}
